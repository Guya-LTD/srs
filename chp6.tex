\chapter{Conclusion and Recommendations}

\section{Conclusion}
The internet has become a major resource in modern business, thus electronic shopping has gained significance not only form the entrepresneur's but also from the customer's point of view. For the entrepreneur, electronic shopping generates new business opportunities and for the customer,it makes comparative shopping possible.

As per a survey, most consumers of online stores are impulsive and usually make a decision to stay on a site within the first few seconds. "Website design is like a shop interior ". If the shop looks poor or like hundreds of other shops the customer is most likely to skip to the other site.

Hence we have designed the project to provide the user with easy navigation, retrieval of data and necessary feedback as much as possible. In this project, the user is provided with an e-commerce web site that can be used to buy any type of products online.

\begin{tabular}{|c|c|}
\hline 
Language & Lines of Code \\ 
\hline 
php & 30 \\ 
\hline 
\end{tabular} 


\section{Recommendation and Future Enhancement}

\subsection{Recommendation}

\subsection{Future Enhancement}
\begin{description}
	\item[Online payment:]
\end{description}
 	
  
\section{Our Conclusions}
After explaining how we have contributed to make this project a reality, we did not want to finish this document without expressing before how this project has contributed in our academic and professional learning. This has been an exceptional year in which we had the opportunity to work in a very dynamic and innovative environment, surrounded by extremely talented and splendid people.

In this atmosphere we were encouraged to try new approaches and technologies to solve design and implementation problems. In fact, it is precisely the acquired technological knowledge we value the most of our learning, especially when compared with our previous experience, which basically consisted of traditional and outdated technologies. This obtained knowledge largely includes the complete functional testing of a web and mobile system, a pending subject in ours professional and academic life.

Working with a proper development environment was also a gratifying experience that we never had the chance to put in practice before. We enjoyed as well getting to know agile methodologies from inside instead of learning the theory from a book. And not only the methodology, but also the work philosophy of the company, which can be simply summarized with the feeling of joining a professional team that cares about doing things right from the start, valuing quality over quantity.

And of course, the development of this template became a very challenging and motivational project to us, with a concept completely different from those we have developed before. It was particularly a pleasure to work with requirements that involved a more humanistic approach of the solution, such as user and developer experience. And yet it was quite surprising how technological these requirements can be.